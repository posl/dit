\section{背景と目的}

\begin{frame}{イメージが抱える課題とその解決手法}
    \begin{multicols}{2}
        \begin{list}{}{\setlength{\itemsep}{1.0zh}}
            \setlength{\itemindent}{-3.0zw}
            \item ビルド時間の短縮
            \item イメージサイズの削減
            \item イメージ分析
            \item サイバー攻撃への対策
            \item 同一性の確認
            \item 再現性の確保
            \item イメージ開発の効率化
            \item 保守性の確保
        \columnbreak
        \setlength{\itemindent}{-4.6zw}
            \begin{onlyenv}<2->
                \item[→] キャッシュの利用,BuildKit
                \item[→] マルチステージビルド,Slim
                \item[→] dive,dlayer
                \item[→] Distrolessイメージ,BuildKit
                \item[→] ハッシュによる確認,イメージ署名
                \item[→] ?
            \end{onlyenv}
            \begin{onlyenv}<2-3>
                \item[→] ?
                \item[→] ?
            \end{onlyenv}
            \begin{onlyenv}<4>
                \item[→] \textcolor{orange}{インタラクティブツール}
                \item[→] \textcolor{orange}{リファクタリング・最適化の機能}
            \end{onlyenv}
        \end{list}
    \end{multicols}

    \begin{onlyenv}<3>
        \begin{tikzpicture}[remember picture]
            \useasboundingbox (0.0, 0.0);
            \begin{scope}[shift={(current page.south west)}]
                \node[callout, callout absolute pointer={(6.4, 1.9)}] at (9.6, 1.5) {提案ツールで解決したい};
                \node[callout, callout absolute pointer={(6.4, 0.9)}] at (9.6, 1.5) {提案ツールで解決したい};
            \end{scope}
        \end{tikzpicture}
    \end{onlyenv}
\end{frame}